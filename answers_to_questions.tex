\section{How important is the problem to practice?}

Problems are the bases of the implementation of practices. Practices are implemented when it actually resolves a problem. Practices without problems to address are useless. Hence, problem is highly significant to the practice.

\section{How will you know if the solution/s that you will achieve would be better than existing ones?}

In order to recognize whether the solutions we achieved are better than existing ones, we need to compare and contrast the output of the prevailing solution with ours. Analysis of the Results and Discussions for both solutions is crucial in determining the dominant answer to the problem.

		\subsection{How will you measure the improvement/s?}
			In order to measure improvements, different tests are considered. In our research, the speed and the quantity of the seed-filled holes are measured. Afterwards, the measurements obtained are compared with the speed and reliability of manual planting. With this, the improvements are measured.
	
		\subsubsection{What is/are your basis/bases for the improvement/s?}
		Improvements are based on the increase in the productivity of a system. It is said that a system has improved when it is capable of yielding better result than prior solutions. In relation with our thesis, increase in speed and number of holes filled were the bases for improvements.

		\subsubsection{Why did you choose that/those basis/bases?}
		These bases are quantifiable such as it can be physically measured. From the physical measurement, comparison are easier on the end of the researchers. Hence, our group decided to choose these bases.
				
		\subsubsection{How significant are your measure/s of the improvement/s?}
		These measures of improvement are highly significant in determining the implementation of the system after the research. When the measure of the improvements are favorable, it is likely that the system is recommended and applied.

\section{What is the difference of the solution/s from existing ones?}

This study was premature, if not introductory, in Philippine agricultural technology targeted specifically with corn production. Current studies did exist but they contributed more on weed removals. Hence, the difference of this study was to focus more on the seeding phase of corn production.

		\subsection{How is it different from previous and existing ones?}

Reiterating, this proposal was targeted on planting corn seeds in the corn production process. Since there were seldom researches about corn planting robots (not to mention that this was more defined in the field of rice planting), there was insufficiency of access to studies about this; presumably. 

\section{What are the assumptions made (that are behind for your proposed solution to work)?}
It is assumed that the soil for planting corn or the corn field itself is already set for the robot to place the seed on, which means, the soil is already bored or holed. It is also assumed that the robot would start correctly on first hole in order to have a successful sequence of seeding, which means, the holes are or assumed to be 5 inches apart from each other and has a diameter of approximately 2 inches, and a depth of 2 inches. The ground is assumed to be flat, and no presence of obstruction or ground level disturbances are present. Lastly, the robot itself is assumed to go on a straight path, and skewing and slipping of the wheels is not present.
	
		\subsection{Will your proposed solution/s be sensitive to these assumptions?}
For this proposal, the system is highly sensitive to these assumptions.
  	\subsection{Can your proposed solution/s be applied to more general cases when some of the assumptions are eliminated? If so, how?}
The system might still work if there are changes in the diameter and depth of the holes in the soil, but must be adjusted if the distance of the holes is inconsistent or not 5 inches apart from each other. In cases where the ground is not stable or flat, or the robot’s wheel skewed or slips, the system fails. 

\section{What is the necessity of your approach / proposed solution/s?}

The necessity of this solution was to integrate the advancement of technology in its wide reach and applicability to various fields. Inarguably, this had been the main purpose of technology. 
	
		\subsection{What will be the limits of applicability of your proposed~solution/s?}
		
		Its limits would have to be the locally available technologies to implement this proposal, as well as the currency of the technologies planned to use (i.e. there might even be a more efficient and more optimized system or method in contrast with the proposed one).
		
		\subsection{What will be the message of the proposed solution to technical people?  How about to non-technical managers and business men?}

Expectedly, this study might be too abstracted (since the systems used were modular) or too inefficient (due to the number of variables taken to study in this study). But, of greater potential to computer vision, these people might begin to stimulate theories of computer vision in this type of application specifically.

Since this study was focused more on the technological side of the industry, these businessmen might think more on gaining profit from these advancements. In return, they would have this marketed to consumers (which included non-technical managers) both international and local; with the hope of the former in aiding the local industry. But, specifically for non-technical managers, they might focus their attention to the potentiality of the proposal to the modernization of the traditional processes that they face; especially considering the current state of technology in the Philippines.
			
\section{How will you know if your proposed solution/s is/are correct?}

The proposed solutions were observed in reference to pertinent benchmarks in the study focus. Hence, with the preliminaries of gathered information from the domain experts, these were met considerably with the outcomes of the study.

		\subsection{Will your results warrant the level of mathematics used (i.e., will the end justify the means)?}

Since this study was a proposal, the fundamental models of statistical analyses were used. And, with only two variables at study, it was but practical to resort to such simple models in order to present the most obvious and most digested relationship of the variables.	    
			
\section{Is/are there an/\_ alternative way/s to get to the same solution/s?}
There are various ways to make the solution work. The solution is not bounded open the system that is proposed on this paper.


		\subsection{Can you come up with illustrating examples, or even better, counter examples to your proposed solution/s?}
For example, the system can implement a different seeding mechanism to sow the seeds and can also integrate a boring mechanism it automatically bore the soil and sow the seed in a single action.
		\subsection{Is there an approximation that can arrive at the essentially the same proposed solution/s more easily?}
		It can be approximated the example solution would work better that the proposed solution because of its ability to bore and seed at the same time, which means, the seeding success rate is next to 100%.
		
\section{If you were the examiner of your proposal, how would you present the proposal in another way?}

In estimation, its method of presentation would be not just in the development of a robot system, but with the consideration of the economical pros and cons of lessening human interaction in corn production. In that way, this proposal is not just a presentation of an idea, but of self-critiquing, as well.

		\subsection{What are the weaknesses of your proposal?}
		
		The weaknesses of this proposal were the timeframe given to execute the study and the projection of its extent against the limitation knowledgebase.