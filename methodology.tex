\section{Implementation}

	The project governed the interfacing of computer vision for a robot system’s navigation. This proposed system was conditioned to navigate on dry land terrains at daytime. Further, certain considerations were taken as per testing. The land area for corn plantation had been downscaled to a 60-inches by 20-inches (by a factor of 1/65th length-wise and 1/197th width-wise) plant box for conduciveness of the study. Such downscaling managed to model one cornrow which was enough to attempt the forwarding control of the robot system. Made out of cardboard boxes lined with garbage bags (for water-proofing), the box frame was filled with loam soil: pre-plowed with two-inches deep as irrigation lines, and pre-holed, five-inches apart.
	
	Proceeding, the mechanism produced for this study was an electrically-driven sower machine. Its cost-effectiveness (since it was made from a recycled canister with a motor screwed to its base) made it implementable to simulate a holed-disc dispensing unit. As the motor was activated manually by the user through a computer, the disc was properly timed to dispense appropriate amount seeds. The process mainly governed on asserting the seeding mechanism first before it proceeded forward.
	
	Mentioning that the land was pre-holed, the robot was now capable to be controlled by the user through a computer. The keyboard keys W (Forward), A (Left), S (Backward) and D (Right) were assigned to direct the basic movement of the system. The E key was made to function to halt the system. And, finally, to consolidate the basic movement of Seed – Forward – Seed – Forward – [...], the 3 key was assigned to operate this iteration. The whole system had been under the implementation of a Raspberry Pi 3 Model B SBC microprocessor unit with Raspberry Pi Camera V2 Video Module as its vision peripheral. With these systems, interfaced and connected to a tank chassis with two motors rated at 0.5A and 6V each; an additional one for the seeding mechanism rated at 0.25A and 5V; back wheels connected directly to the motor, leaving the front wheels as free wheels. Supplied by two batteries (one per motor) rated at 5.7V with 5780mA current delivery each.

	
\section{Evaluation}
	The robot system was expected to plant 11 holes in 1.286 minutes per row. Since this data were to include the seeding and the holing processes, the group attempted to modify the system by removing the puncturing mechanism due to time constrains. With these benchmarks scaled from a 100-meter-by-100-meter land area with eight persons to labor the whole field, the system relatively delivered to emulate the benchmarked performance. With three seeds to be planted per hole set for twenty trials (in ideal), the gathered data managed to reach an observable consistency in detecting the possibility of jamming. This was the very hampering concern of the whole study. Seeds were stuck inside the crevices of the motor with the base of the canister. At this point, the group decided to remove the factor of amount of seeds and just went with the success of having at least a seed put to a hole. The factors affecting the prior trials had been the following:

\begin{itemize}
\item {Misaligned seeding machine}
\item {Stillness or non-rotation of the seeding machine}
\item {Wrong dispensing of seeds (with less than 3 seeds)}
\item {Positioning of the chassis in-line with the irrigation line}
\item {Unlevelled land area where the belt wheel got jammed}
\end{itemize}

\section{Summary}
	In a nutshell, the whole study was a downscale simulation of a corn field plantation seen in the Philippines. With a system made out of Raspberry Pi system and a homemade seeding mechanism only, the system was set to plant three corn seeds in an 11-hole cornrows in under a minute. Through the implementation of a computer vision with manual navigation controls, the study delivered pleasing results.
